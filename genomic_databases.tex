\title{A Very Simple \LaTeXe{} Template}
\author{
        Vitaly Surazhsky \\
                Department of Computer Science\\
        Technion---Israel Institute of Technology\\
        Technion City, Haifa 32000, \underline{Israel}
            \and
        Yossi Gil\\
        Department of Computer Science\\
        Technion---Israel Institute of Technology\\
        Technion City, Haifa 32000, \underline{Israel}
}
\date{\today}

\documentclass[12pt]{article}

\usepackage{mathtools}
\begin{document}
\maketitle

% \begin{abstract}
% This is the paper's abstract \ldots
% \end{abstract}

\section*{Introduction}
Haplotype database representation

\subsection*{Objective}
We want to apply differential privacy Techniques to haplotypes

\subsection*{Basics}
A database $x$ is a collection  of elements of a universe $\mathcal{X}$ of rows (records)
The histogram of a database is a vector $x_i \ldots x_n, n = |\mathcal{X}| $and $x_i$ is the number
of repeats of a row in the database. The $l_1$ norm of a database is defined as
\begin{equation}
        \|x\|_1 = \sum_{i=1}^{|\mathcal{X}|} |x_i|
\end{equation}



The distance is defined thought the norm in the usual way

A haplotype is a  collection of nodes and edges (connections between nodes), in
other words a graph database.
Graph databases can be represented relational through with adjacency lists: https://en.wikipedia.org/wiki/Adjacency_list
So for any graph we need a column of nodes, and a column of adjacent nodes

\begin{corollary*}
Let x,y be haplotypes. Then $\|x-y\|_1 \leq 1$ iff x and y differ on exacly one node

\end{corollary*}
\paragraph{Outline}
The remainder of this article is organized as follows.
Section~\ref{previous work} gives account of previous work.
Our new and exciting results are described in Section~\ref{results}.
Finally, Section~\ref{conclusions} gives the conclusions.


\bibliographystyle{abbrv}
\bibliography{simple}

\end{document}
This is never printed
