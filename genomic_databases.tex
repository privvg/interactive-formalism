\title{Privacy pangenomics}
\author{
        Christos Chatzifountas
}
\date{\today}

\documentclass[12pt]{article}

\usepackage[utf8]{inputenc}
\usepackage[english]{babel}

\usepackage{mathtools}
\usepackage{amssymb,amsmath,amsthm}

\usepackage{hyperref}

\newtheorem{lemma}{Lemma}
\begin{document}
\maketitle


% \begin{abstract}
% This is the paper's abstract \ldots
% \end{abstract}

\section*{Introduction}
Haplotype database representation

\subsection*{Objective}
We want to apply differential privacy Techniques to haplotypes

\subsection*{Basics}
A database $x$ is a collection  of elements of a universe $\mathcal{X}$ of rows (records)
The histogram of a database is a vector $x_i \ldots x_n, n = |\mathcal{X}| $and $x_i$ is the number
of repeats of a row in the database. The $l_1$ norm of a database is defined in thought of
it's histogram:
\begin{equation}
        \|x\|_1 = \sum_{i=1}^{|\mathcal{X}|} |x_i|
\end{equation}



The distance is defined thought the norm in the usual way

A haplotype is a  collection of nodes and edges (connections between nodes), in
other words a graph database.
Graph databases can be represented relational through with \href{ https://en.wikipedia.org/wiki/Adjacency_list}{adjacency lists}
So for any graph we need a column of nodes, and a column of adjacent nodes

\begin{lemma}
Let x,y be haplotypes. If they differ in a node then  $\|x-y\|_1 \geq 1$ %iff either x has one more edge than y or y has one more edge than x  with no other adjacent nodes
\end{lemma}


\begin{proof}
Let x,y  be haplotypes that on one node or more. Then that node and the adjacency differ in their adjacency list representation. Hence their
histograms differ on two or more coordinates. If two integers are different, their difference is greater than one , hence $\|x-y\| \geq |x_n-y_n| + |x_m-y_m| \geq 2$
\end{proof}



% \bibliographystyle{abbrv}
\bibliography{simple}

\end{document}
This is never printed
